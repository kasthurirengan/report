The later half of the 20th century had seen a sea change in field of Integrated Circuits. The complexity of design had increased drastically from few hundred to billions of gates as of now. This made the conventional method of testing and simulation to be incomplete as one can never test the complete functionality of a system. This led to the idea of using Formal Verification in the field of Integrated Circuits."Formal verification is the act of proving or disproving the correctness of intended algorithms underlying a system with respect to a certain formal specification or property, using formal methods of mathematics"\cite{Alok 2010}. Formal verification provides exhaustive exploration of all possible behaviours rather than the traditional approach of simulation/testing which explore only few possible behaviours of a system. There are several methods of formal verification available out of which we will be looking into a method of model checking. The main advantages of the method is,
\begin{itemize}
\item The ability to perform verification in a completely automatic manner without any intervention from the user.
\item The result of model checking is always either True or False
\item If the property is failed to satisfy, it always produces a counterexample which provides an insight about the failure of the system
\end{itemize}