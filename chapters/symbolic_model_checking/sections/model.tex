We use Kripke structure for modelling our system. Kripke structure is a tuple M = (S,R,I,AP,L).
\begin{itemize}
\item $S$ is set of states
\item $R\subseteq S\times S$ is the transition relation
\item $I\subseteq S$ is the set of possible initial states
\item $AP$ is a set of atomic propositions
\item $L:S\rightarrow 2^{AP}$ is a labeling function: each state is labeled with the\\ atomic propositions that are true in that state
\end{itemize}
S is also called as the state space of the system.

\begin{figure}
\centering
\begin{tikzpicture}[->,>=stealth',shorten >=1pt,auto,node distance=4cm,
  thick,main node/.style={circle,fill=blue!20,draw,font=\sffamily\Large\bfseries}]

   \node[main node] (2)  {000};
  \node[main node] (3) [above of=2] {001};
  \node[main node] (5) [above of=3] {011};
  \node[main node] (4) [above of=5] {010};
  \node[main node] (6) [right of=4] {100};
  \node[main node] (7) [right of=5] {101};
  \node[main node] (8) [right of=2] {110};
  \node[main node] (9) [right of=3] {111};


  \path[every node/.style={font=\sffamily\small}]
    (2) edge [bend left]  (4)
        edge [bend left]  (5)
    (3) edge[loop left] (3)
        edge   (2)
    (4) edge (6)
          edge (7)
    (5)edge (3)
         edge[bend left] (2)
    (6)edge [bend left] (8)
         edge[bend left] (9)
    (7)edge (2)
         edge (3)
    (8) edge (2)
         edge(3)
    (9) edge (2)
          edge(3)
         ;
\end{tikzpicture}
\caption{Kripke structure of a 2-bit Counter} 
\label{fig:kripke structure 2-bit counter}
\end{figure}

Figure \ref{fig:kripke structure 2-bit counter} is a kripke structure of the figure \ref{fig:2-bit Counter}. Here state 000 corresponds to $\neg v1 \neg v0 \neg rst$.

Till early 1980's, transition relation of a system was represented using adjacency list\cite{Clarke 2008}.  Due to the increase in the number of states, state transition graph became too large to handle using adjacency list. \cite{McMillian 1993} suggested a novel approach to solve the state exploration problem by using a symbolic representation of the state transition graph. It was based on ROBDD. For a given order, ROBDD of a boolean expression is always canonical\cite{Bryant 1986}.