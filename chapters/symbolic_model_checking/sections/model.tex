We use Kripke structure for modelling our system. Kripke structure is a tuple M = (S,R,I,AP,L).
\begin{itemize}
\item $S$ is set of states
\item $R\subseteq S\times S$ is the transition relation
\item $I\subseteq S$ is the set of possible initial states
\item $AP$ is a set of atomic propositions
\item $L:S\rightarrow 2^{AP}$ is a labeling function: each state is labeled with the\\ atomic propositions that are true in that state
\end{itemize}
S is also called as the state space of the system.
\newline

\begin{figure}
\centering
\begin{tikzpicture}[->,>=stealth',shorten >=1pt,auto,node distance=4cm,
  thick,main node/.style={circle,fill=blue!20,draw,font=\sffamily\Large\bfseries}]

  \node[main node] (1)  {S0};
  \node[main node] (2) [right of=1] {S1};
  \node[main node] (3) [below of=2] {S2};
  \node[main node] (4) [below of=1] {S3};

  \path[every node/.style={font=\sffamily\small}]
    (1) edge (2)
        edge [bend left] (3)
    (2) edge (3)
    (3) edge (4)
        edge [bend left] (1)
    (4) edge (1)
         edge[loop left] (4)
         ;
\end{tikzpicture}
\caption{Kripke structure} \label{kripke structure}
\end{figure}


The Figure ~\ref{kripke structure} is a kripke structure. S0,S1,S2 and S3 are the states of the system. \{S0,S1\},\{S0,S2\},\{S3,S3\}.. are the transitions of the system.
\newline

Till early 1980's, transition relation of a system was represented using adjacency list\cite{Clarke 2008}.  Due to the increase in the number of states, state transition graph became too large to handle using adjacency list. \cite{McMillian 1993} suggested a novel approach to solve the state exploration problem by using a symbolic representation of the state transition graph. It was based on ROBDD. For a given order, ROBDD of a boolean expression is always canonical\cite{Bryant 1986}.