Boolean expressions can be represented using a rooted, direct, acyclic graph called the BDD. BDD was first introduced by Lee\cite{Lee 1959}. An efficient data structure for representing the BDD and efficient algorithms for manipulating BDD's were later developed by \cite{Bryant 1986}. He also introduced the concept of ordering of variables in a BDD, which would affect the size significantly. 

Let's see few important expressions for BDD manipulations.

\subsubsection*{Shannon Expansion}
It is defined as the expansion of a boolean formula with respect to a boolean variable. $f|_{x\leftarrow 0}$ and $f|_{x_\leftarrow 1}$ are called the positive and negative Shannon co-factors of  boolean formula f with respect to boolean variable x.
\begin{equation}
\label{shannon}
f=(\neg x\land f|_{x\leftarrow 0}) \lor (x\land f|_{x\leftarrow 1})
\end{equation}

\subsubsection*{Existential quantification}
If either the positive co-factor or the negative co-factor of a boolean expression with respect to a boolean variable x is true, then the boolean variable x can be existentially quantified out.
\begin{equation}
\label{eqn:EQ}
\exists_xf=f|_{x\leftarrow 0} \lor f|_{x\leftarrow 1}
\end{equation}


\subsubsection*{Universal quantification}
Only if both the positive and negative co-factor of a boolean expression with respect to a boolean variable x is true, then the boolean variable x can be universally quantified out. 
\begin{equation}
\label{UQ}
\forall_xf=f|_{x\leftarrow 0} \land f|_{x\leftarrow 1}
\end{equation}

\subsubsection*{BDD Operations}
Boolean operations like AND, OR, NOT etc.. can be implemented as algorithms on OBDD's. The algorithm takes OBDD's as input and produce a reduced form of OBDD's\cite{Bryant 1986}. The variable ordering of the original OBDD's are preserved. The reduction of OBDD is based on the following reduction rules.

\begin{itemize}
\item Any node having identical child nodes are removed.
\item 2 Nodes with isomorphic BDD's are removed.
\end{itemize}

ROBDD is the most compact representation of BDD by using the above mentioned reduction rule which results in eliminating nodes which in turn reduces the memory used to construct ROBDD. 
\begin{equation}
f<op>g = (\neg x \land (f|_{x\leftarrow 0} <op> g|_{x\leftarrow 0})) \lor (x \land (f|_{x\leftarrow 1} <op> g|_{x\leftarrow 1}))
\label{eqn:op}
\end{equation}

Equation ~\ref{eqn:op} can be recursively called for computing the OBDD representation of $f<op>g$. Reduction rules can be applied alongside the operation thereby generating a ROBDD.